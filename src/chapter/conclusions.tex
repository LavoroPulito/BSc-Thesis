\chapter*{Conclusions}
In this thesis, we have outlined all the key intermediate steps that lead from the Edmonds-Karp algorithm to Orlin's algorithm. A crucial point that emerged is the strong interconnection between the various algorithms.

Although each algorithm significantly enhances the efficiency of the previous one, the core principles behind the solutions are not radically different. For instance, we observed that Dinitz’s algorithm modifies only the sequence of the augmenting phase and the BFS computation found in the Edmonds-Karp algorithm. In turn, the Goldberg-Rao algorithm builds upon Dinitz’s work, executing the augmentation phases on a more contracted graph.

Orlin’s algorithm further capitalizes on the Goldberg-Rao method, using it in its original form to compute an $\alpha$-optimal flow on a specially compacted graph, designed to reduce computational costs even further.

For graphs that are even sparser ($n = O(m)$), Orlin proposed an alternative approach. In this version, given the number of edges, the procedure for dynamic transitive closure (one of the main bottlenecks of the algorithm) can be replaced with a more efficient method, thereby reducing the computational cost by a logarithmic factor, bringing it down to $O(n^2/\log n)$.

Further advances in this field were made in 2018, when Orlin, together with Xiao-Yue Gong, introduced an algorithm\cite{orlin2019fastmaxflowalgorithm} that strictly outperforms the King et al. algorithm, particularly in cases where $m = o(n\log n)$, surpassing it by a factor of $\log\log n$.
\addcontentsline{toc}{chapter}{Conclusions}

